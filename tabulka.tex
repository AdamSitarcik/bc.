\documentclass[12pt, oneside]{book}
\usepackage[T1]{fontenc}
\usepackage{lmodern}
\usepackage{ucs}
\usepackage[utf8]{inputenc}
\usepackage[slovak]{babel}
\newcommand\sktxt[1]{\foreignlanguage{slovak}{#1}}
\usepackage{a4wide}
\usepackage{amsmath,amsfonts,amssymb}
\usepackage{setspace}
\usepackage{MnSymbol,wasysym}
\usepackage{multirow,tabularx}
\usepackage{lscape}
\usepackage{pbox}
\usepackage{tabu}
\usepackage[font=small,labelfont=bf]{caption}
\usepackage{url}
\usepackage[hyphenbreaks]{breakurl}
\usepackage{hyperref}
\usepackage{blindtext}
\usepackage[version=4]{mhchem}
\usepackage{graphicx}
\usepackage{tikz}
\usepackage{hyperref}
\usepackage{pdfpages}
\usepackage{fancyhdr}
\usepackage{multicol}
\usepackage[a4paper,top=2.5cm,bottom=2.5cm,left=3.5cm,right=2cm]{geometry}
\usepackage{mathtools}
\usepackage{caption}

\begin{document}

\subsection*{$^{260} \mathrm{Sg}$} 
\vspace{5mm}
Tabuľka  \ref{tab:hl_hm_mod_260sg} ukazuje teoretické polčasy rozpadu vypočítané modelom  \hyperref[sec:semfis]{SemFIS}, pre $Q$ hodnoty z tabuľky \ref{tab:260sg_q}.

\begin{table}[h!]
	\centering
	\caption{Porovnanie teoretických polčasov rozpadu reťazca $^{260} \mathrm{Sg} \xrightarrow{4 \alpha} {^{244} \mathrm{Cf}}$, určených podľa modelu z \cite{poe80} (upraveným v \cite{poe06}), pre $Q$ hodnoty určené z teoretických hmotnostných modelov podľa \ref{eq:q_ubytok}.}
	\label{tab:hl_hm_mod_260sg}
	\resizebox{\textwidth}{!}{\begin{tabular}{lcccccc}
    \hline \hline
& $T(Q_{exp})[s]$ & $T(Q_{FRDM})[s]$ & $T(Q_{SSME})[s]$ & $T(Q_{ETFSI})[s]$ & $T(Q_{TFM})[s]$ & $T(Q_{S\&S})[s]$ \\ \hline 
$^{260} \mathrm{Sg}$  & 0.0068                & 0.0056          & 0.0074          & 0.69             & 0.0018          & 0.0044  \\
$^{256} \mathrm{Rf}$ & 0.797                  & 2.75             & 0.12            & 0.021            & 0.76           & 0.25           \\
$^{252} \mathrm{No}$ & 2.43                  & 11.3             & 2.99             & 0.12             & 3.39            & 2.35           \\
$^{248} \mathrm{Fm}$ & 31.34                  & 644              & 131              & 1.73             & 165             & 61.7           \\ 
    \hline \hline
	\end{tabular}}
\end{table}

Grafické porovnanie polčasov z tabuľky \ref{tab:hl_hm_mod_260sg} je na obrázku \ref{fig:porovnanie_halflife_260sg}.

\begin{figure}[h!]
	\begin{center}
	\includegraphics[width=0.8\textwidth]{images/porovnanie_halflife_260sg.pdf}
	\caption{Rovnako ako na obrázku \ref{fig:porovnanie_halflife_259sg} ale pre reťazec $^{260} \mathrm{Sg} \xrightarrow{4 \alpha} {^{244} \mathrm{Cf}}$.}
	\label{fig:porovnanie_halflife_260sg}
    \end{center}
\end{figure}

Podobne, ako pre $^{259} \mathrm{Sg}$, aj pri $^{260} \mathrm{Sg}$ sú odchýlky prechádzajúcich jadier rádovo do 1.5 rádu, pre $^{260} \mathrm{Sg}$ sú odchýlky minimálne, menej ako pol rádu, okrem polčasu určeného z $Q$ hodnoty podľa modelu ETFSI, tu je odchýlka viac než dva rády.

\subsection*{$^{261} \mathrm{Sg}$} 
\vspace{5mm}
Tabuľka  \ref{tab:hl_hm_mod_261sg} ukazuje teoretické polčasy rozpadu vypočítané modelom  \hyperref[sec:semfis]{SemFIS}, pre $Q$ hodnoty z tabuľky \ref{tab:261sg_q}.

\begin{table}[h!]
	\centering
	\caption{Porovnanie teoretických polčasov rozpadu reťazca $^{261} \mathrm{Sg} \xrightarrow{4 \alpha} {^{245} \mathrm{Cf}}$, určených podľa modelu z \cite{poe80} (upraveným v \cite{poe06}), pre $Q$ hodnoty určené z teoretických hmotnostných modelov podľa \ref{eq:q_ubytok}.}
	\label{tab:hl_hm_mod_261sg}
	\resizebox{\textwidth}{!}{\begin{tabular}{lccccc}
    \hline \hline
& $T(Q_{exp})[s]$ & $T(Q_{FRDM})[s]$ & $T(Q_{SSME})[s]$ & $T(Q_{ETFSI})[s]$ & $T(Q_{TFM})[s]$ \\ \hline 
$^{261} \mathrm{Sg}$ & 0.079 & 0.13  & 0.054 & 12.7  & 0.041 \\
$^{257} \mathrm{Rf}$ & 0.89  & 0.81  & 0.85  & 0.035 & 0.24  \\
$^{253} \mathrm{No}$ & 21    & 118   & 22    & 1.42  & 33    \\
$^{249} \mathrm{Fm}$ & 1028  & 12500 & 1090  & 21    & 3200  \\ 
    \hline \hline
	\end{tabular}}
\end{table}

Grafické porovnanie polčasov z tabuľky \ref{tab:hl_hm_mod_261sg} je na obrázku \ref{fig:porovnanie_halflife_261sg}.

\begin{figure}[h!]
	\begin{center}
	\includegraphics[width=0.8\textwidth]{images/porovnanie_halflife_261sg.pdf}
	\caption{Rovnako ako na obrázku \ref{fig:porovnanie_halflife_259sg} ale pre reťazec $^{261} \mathrm{Sg} \xrightarrow{4 \alpha} {^{245} \mathrm{Cf}}$.}
	\label{fig:porovnanie_halflife_261sg}
    \end{center}
\end{figure}

Na rozdiel od $^{260} \mathrm{Sg}$ a $^{259} \mathrm{Sg}$, sú odchýlky polčasov pre finálne jadro rozpadového reťazca, $^{261} \mathrm{Sg}$, veľké, pre model TFM a ETFSI vyše dvoch rádov, rovnako aj polčasy ostatných jadier sú odchýlené od teoretickej hodnoty o približné rád. Napriek tomu, že model SSM pre hmotnosti tohto rozpadového reťazca sedel najlepšie, pri určení polčasu sa tiež ukázali odchýlky od experimentálnych hodnôt.

\subsection{Emisia klastrov}
Aplikáciu teoretických modelov polčasov rozpadu sme rozšírili tiež o emisiu klastrov. V tabuľke \ref{tab:cluster} sú uvedené logaritmy polčasov rozpadu pre jadrá rozpadajúce sa týmto typom rozpadu. 
\\
\\
Grafické porovnanie experimentálnych a teoretických polčasov rozpadu klastrovej emisie je na obrázku \ref{fig:cluster_hl}. Z neho vidno, že jednotlivé modely sa od experimentálnych hodnôt líšia  najviac o dva rády, čo je s ohľadom na celkovo dlhý polčas rozpadu (rádovo $10^{20}~\mathrm{s}$) dobrá presnosť. 
\\
\\
Zaujímavé je, že napriek tomu, aká vzácna a málo preskúmaná je emisia klastrov, dané teoretické modely dosahujú často lepšiu zhodu s experimentálnymi hodnotami pri aplikácii práve na tento rozpad, ako pri aplikácii na $\alhpa$ rozpad.


\begin{figure}[h!]
	\begin{center}
	\includegraphics[width=\textwidth]{images/cluster_hl}
	\caption{V obrázku sú materské jadrá, ich módy rozpadu sú v tabuľke \ref{tab:cluster}. Modely: Poe06 - z \cite{poe06}; Qi2, Qi3 - \hyperref[sec:qi_udl]{UDL}. Koeficienty podľa  Tabuľky \ref{tab:qi}: II~-~emisia klastrov (Qi2); III - zmiešané pre $\alpha$ rozpad/emisiu klastrov (Qi3). y-ová os predstavuje logaritmus pomeru experimentálneho a teoretického polčasu rozpadu - faktor potlačenia ($HF$).}
	\label{fig:cluster_hl}
    \end{center}
\end{figure}

\subsection{Protónová emisia}
Ďalšie rozšírenie aplikácií predstavuje protónová emisia. Polčasy tohto rozpadu porovnané s experimentálnymi polčasmi sú na obrázku \ref{fig:proton_hl}.
\\
\\
Z Obrázku \ref{fig:proton_hl} je badateľný prudký pokles $HF$ medzi $^{155} \mathrm{Ta}$ a $^{160} \mathrm{Re}$. Tento pokles môže byť spôsobený efektom uzatvorenej neutrónovej pre $N=82$. Napriek tomu, že protónová emisia mení len protónové číslo jadra, zrejme neutrónové uzatvorené vrstvy tiež zvyšujú stabilitu takýchto jadier.
\\
\\
Modely sme aplikovali aj na možnú protónovú emisiu jadier $^{197} \mathrm{Fr}$ a $^{199}\mathrm{Fr}$. Tieto jadrá boli bližšie skúmané v \cite{kalaninova}, kde jadro $^{197} \mathrm{Fr}$ bolo pozorované cez jeden rozpadový $\alpha$ reťazec a pre jadro $^{199}\mathrm{Fr}$ bolo zaznamenaných \sim 40 $\alpha$ rozpadov. Je dôvod predpokladať, že tieto jadrá sa môžu rozpadať aj emisiou protónu, kvôli nepárnemu protónovému číslu a polohe blízko protónového driplinu.
\\
\\
Vetviace pomery týchto jadier sme odhadli na základe pozorovaných rozpadov. Pre $^{197}\mathrm{Fr}$ sme ho určili ako $b_p=1$ a pre $^{199}\mathrm{Fr}$ ako $b_p=1/40$. Tieto pomery sú však limitované počtom pozorovaných rozpadov, protónová emisia (ak sa naozaj vyskytne) môže nastať aj po niekoľko násobne väčšom počte $\alpha$ premien. Preto dané $\log HF$ v obrázku sú len \textbf{dolné limity}, skutočné vetviace pomery týchto jadier môžu byť omnoho menšie - experimentálne polčasy rozpadov budú vyššie.
\\
\\
Preto podobný pokles ako medzi $^{155} \mathrm{Ta}$ a $^{160} \mathrm{Re}$ (pravdepodobne spôsobený uzavretou neutrónovou vrstvou) môže, ale aj nemusí byť spôsobený uzatvorenou protónovou vrstvou.
\\
\\
Celkovo sa dá vidieť určitý systematický posun modelu Qi2 oproti ostatným dvom modelom (3-4 rády). Tento posun môže byť spôsobený tým, že koeficienty II z \ref{tab:qi} sú určené pre emisiu klastrov, t.j. jadrá  násobne väčšie ako je $\alpha$ častica. Preto pri aplikovaní na emisiu protónu dochádza k väčšej odchýlke oproti Qi3, ktorý je určený aj pre $\alpha$ rozpad.
\vspace{6mm}
\begin{figure}[h!]
    	\begin{center}
	\includegraphics[width=0.8\textwidth]{images/proton_hl}
	\caption{Porovnanie experimentálnych a teoretických polčasov rozpadu protónovej emisie. x-ová os predstavuje protónové číslo skúmaného jadra, y-ová zas logaritmus faktoru potlačenia ($\log{HF}$). Experimentálne hodnoty z \cite{audi2012ame2012}.}
	\label{fig:proton_hl}
    \end{center}
\end{figure}

\begin{table}[h!]
\centering
\caption{Modely: (I) - \hyperref[sec:poe06]{Poenaru 2006}; (II)~-~\hyperref[sec:qi_udl]{UDL} , koeficienty pre klastre; (III)~-~\hyperref[sec:qi_udl]{UDl}, koeficienty zmiešané pre $\alpha$ rozpad aj emisiu zložených jadier. Experimentálne hodnoty polčasov rozpadu a $Q$ hodnôt sú z \cite{santhosh2012cluster}. Neistoty nie sú v obrázku uvedené, z dôvodu absencie v referenčnom článku.}
\label{tab:cluster}
\vspace{5mm}
\begin{tabular}{cccccc}
\hline
\hline
Izotop      & \begin{tabular}[c]{@{}l@{}}Emitované\\ jadro\end{tabular} & $\log T_{1/2} ^{exp.}$ & $\log T_{1/2} (I)$ & $\log T_{1/2} (II)$ & $\log  T_{1/2} (III)$ \\ 
\hline
$^{221}\mathrm{Fr}$ & $^{14}\mathrm{C}$  & 14.52 & 14.24 & 13.32 & 15.43 \\
$^{221}\mathrm{Ra}$ & $^{14}\mathrm{C}$  & 13.39 & 13.13 & 12.26 & 14.27 \\
$^{222}\mathrm{Ra}$ & $^{14}\mathrm{C}$  & 11.01 & 11.95 & 11.07 & 12.99 \\
$^{223}\mathrm{Ra}$ & $^{14}\mathrm{C}$  & 15.06 & 13.99 & 13.16 & 15.23 \\
$^{224}\mathrm{Ra}$ & $^{14}\mathrm{C}$  & 15.86 & 16.4  & 15.58 & 17.83 \\
$^{225}\mathrm{Ac}$ & $^{14}\mathrm{C}$  & 17.16 & 17.39 & 16.64 & 18.95 \\
$^{226}\mathrm{Ra}$ & $^{14}\mathrm{C}$  & 21.19 & 21.2  & 20.33 & 22.94 \\
$^{228}\mathrm{Th}$ & $^{20}\mathrm{O}$  & 20.72 & 21.9  & 21.56 & 22.95 \\
$^{230}\mathrm{U}$  & $^{22}\mathrm{Ne}$ & 19.57 & 20.2 & 20.74 & 21.24 \\
$^{230}\mathrm{Th}$ & $^{24}\mathrm{Ne}$ & 24.61 & 24.65 & 24.71 & 25.35 \\
$^{231}\mathrm{Pa}$ & $^{24}\mathrm{Ne}$ & 23.23 & 22.13 & 22.09 & 22.48 \\
$^{232}\mathrm{U}$  & $^{24}\mathrm{Ne}$ & 21.08 & 20.76 & 20.68  & 20.93 \\
$^{234}\mathrm{U}$  & $^{24}\mathrm{Ne}$ & 25.92 & 25.38 & 25.73 & 26.35 \\
$^{233}\mathrm{U}$  & $^{25}\mathrm{Ne}$ & 24.83 & 23.69 & 23.55 & 23.89 \\
$^{236}\mathrm{Pu}$ & $^{28}\mathrm{Mg}$ & 21.67 & 21.19 & 21.02 & 20.23 \\
$^{238}\mathrm{Pu}$ & $^{28}\mathrm{Mg}$ & 25.7  & 25.33 & 25.78 & 25.34 \\
$^{236}\mathrm{U}$  & $^{30}\mathrm{Mg}$ & 27.58 & 28.78 & 28.62 & 28.27 \\
$^{237}\mathrm{Np}$ & $^{30}\mathrm{Mg}$ & 26.93 & 26.88 & 26.61 & 26.05 \\
$^{240}\mathrm{Pu}$ & $^{34}\mathrm{Si}$ & 25.52 & 27.12 & 26.42 & 24.86 \\
$^{242}\mathrm{Cm}$ & $^{34}\mathrm{Si}$ & 23.15 & 23.93 & 22.85 & 20.9 \\
\hline \hline
\end{tabular}
\end{table}
